\newpage

\section{Risks}
\label{sec:risks}

\subsection{User Acceptance}

One risk of implementing Social Login services on a website is user acceptance.
According to a study in 2014, 85\% of people were aware of social login features while 59.4\% were already
using them.\footcite[Cp. ][p. 68]{Gafni2014}
This number might have gone up in the mean time, as social login has become much more present across the web.
A study from 2018 showed, that acceptance of social login varies within age groups and knowledge levels.
Male users seem to have a higher acceptance than female users and people with knowledge about privacy related topics
are more careful and use them less.\footcite[Cp.][p. 7]{Jiang2018}
Additionally, some users are already used to the username and password system and
might be hesitant to change their behaviours. This is supported by the password management and cloud-sync features
of most modern browsers, which allow users to automatically generate and fill in passwords.\footcite[Cp.][p. 22]{Bazaz2016}

Users might also not trust the social login provider to keep their personal data safe.
They fear that personal information saved within their social media account is being shared with third-parties without
their consent and that they thus lose their anonymity on the internet.\footcite[Cp.][p. 22]{Bazaz2016}\footcite[Cp.][p.61]{Gafni2014}
Users are also concerned that \ac{IdP} track them on the websites they are logged in on and might be able to collect
an extensive range of behaviour and user data.\footcite[Cp.][p. 60]{Gafni2014}

Lastly, being redirected to a different website and asked to log in there can be confusing for less technically literate
users. As the \ac{SSO} process requires this, users might get confused and shy away from these services.\footcite[Cp.][p. 23]{Bazaz2016}

Because of the reasons mentioned above, if a website decides to rely solely on \ac{SSO} for user logins,
specific groups of users are inherently less likely to register with the website, which leads to a loss of potential customers.

\subsection{Identity Provider as Single Point of Failure}

Users that created an account using \ac{SSO} typically do not create a seperate password with their account,
as authentication is handled by the \ac{IdP}. The only way for a user to sign in is through the \ac{IdP}.\footcite[Cp.][p. 24]{Bazaz2016}
This means that a given user would not be able to log in at all in the following scenarios:

\begin{itemize}
    \item The \ac{IdP} encounters an outage and their authentication service is unreachable.
    \item The \ac{IdP} encounters data loss and is not able to provide the necessary identity information.
    \item The \ac{IdP} terminates their service indefinitely.
\end{itemize}

While the first scenario would only result in a temporary inability to sign in, the others could result
in a permanent loss of all users that signed in using that specific \ac{IdP}.\footcite[Cp.][p. 22]{Bazaz2016}
This means that the \ac{SP} is heavily dependent on the \ac{IdP} to have uninterrupted availability and to never terminate
their service\footcite[Cp.][p. 62]{Sun2010}.
There are ways to recover these accounts (e.g. e-mail account reset), but this only works if the \ac{SP} has stored
the appropriate user information on their own server, which might not always be the case.

\subsection{Security of SSO Technologies}

Researchers have found security flaws in many widely used \ac{SSO} implementations.
This includes OpenID but also specific implementations from Google and Facebook.\footcite[Cp.][p. 376]{Wang2012}
Although such vulnerabilities usually get patched very quickly\footcite[Cp.][p. 370-375]{Wang2012}, this might still give hackers the
opportunity to attain the identity of a different user and act in their name.
Most of the vulnerabilities stemmed from problems on the side of the \ac{SP},
which used the given \ac{SSO} protocol in a way that the \ac{IdP} did not anticipate\footcite[Cp.][p. 376]{Wang2012}.
Security issues like these are very hard to recognize as they depend on the 
specific implementation on each website.
The security benefits of not having to store passwords are thus negated by the possibility
of having an insecure \ac{SSO} implementation.

\subsection{Security Standards of Identity Providers}

The complete reliance on an \ac{IdP} for authentication also has implications for security.
The \ac{IdP} gets to choose which security guidelines they implement, what password criteria they define,
whether or not they require the use of \ac{MFA} and how their password databases are secured.
If the \ac{IdP} relies on bad security practices, this directly reflects on the security of the \ac{SP}.
Hackers taking over a user's account with the \ac{IdP} leads to them also getting access to the user's data on the \ac{SP}'s site.

Making this worse, \acp{SP} have a limited amount of flexibility. If a percentage of their userbase has signed in through an \ac{SSO}
provider, it is not possible to switch off of that provider without losing those users.
This further strengthens the dependency on \acp{IdP}.

% Legal requirements: https://dl.acm.org/doi/pdf/10.1145/3167132.3167259?casa_token=Fvk10TNXWLoAAAAA:xsPxmQJxFKbd4GzAerqUvjZ8JJfgTfzYhql9zZ2Zy6aIKsj2sfjmt0f-rJ_t7RfxQm5e1AqkS619-Q
% https://dl.acm.org/doi/pdf/10.1145/2554850.2554857?casa_token=N737urQVZT8AAAAA:etyumP7dyJ5-sSpmqwMXCJMXH3Ib7Tl0x09xLS8yiDqPYoulz3cUJyRFW0pYQkwn26tl28__Dcs_jQ