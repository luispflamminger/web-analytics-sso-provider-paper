\newpage
\section{Opportunities}
\label{sec:opportunities}

\subsection{Increasing Conversion Rates}

Conversion rate is one of the most important e-commerce metrics.
It gives the percentage of users who visited the website that resulted in actual sales.\footcite[Cp.][p. 1]{Gabir2018}
There are a number of factors that can influence conversion rate.
On the one hand these include multiple aspects that have nothing to do with the website itself.
The type and range of products offered on the website are important.
A store selling clothing might have a much higher conversion rate than a website selling luxury cars, for example\footcite[Cp.][p.165]{Fatta2018}.
The pricing strategy also has an impact. If the website is offering attractive pricing on their products and additional
bonuses like free shipping or frequent discounts, conversion rates are increased\footcite[Cp.][p.165]{Fatta2018}.
The promotion aspect is another important factor. If the brand's advertising clearly and effectively promotes its products,
visitors are more likely to buy on the first visit\footcite[Cp.][p.165]{Fatta2018}.

The factors mentioned above are related to the web presence, in that offers like discounts and free shipping
need to be effectively presented on the platform. But in order to have a decent conversion rate, technical
factors on the website also have to be addressed\footcite[Cp.][p. 5]{Gabir2018}.

Usability is directly related to how the website is built and which features the e-commerce system has\footcite[Cp.][p. 2f]{Gabir2018}.
There are three dimensions of usability on a website.
Information Quality describes how accurate, relevant and timely the content on the website is.
Service Quality describes how interactive and responsive the website is, how good the search function works
and how well the security and privacy policies are presented.
System Quality includes how easy the site is to navigate, how fast the checkout process is
and aspects like accessibility, and consistency of layout.\footcite[Cp. ][p. 3]{Kuan}

Research has shown, that all three dimensions have a direct impact on conversion rates and customer retention.
Users that perceive a higher information and system quality are significantly more likely to complete
a purchase\footcite[Cp.][p. 6]{Kuan}.
Service quality might have a measurable impact, but studies diverge here\footcite[Cp.][p. 6]{Kuan}\footcite[Cp.][p. 5]{Gabir2018}.
A survey shows that the second most common reason for adults in the \ac{US} not completing the checkout
process is, that they had to create an account. 17\% said, that the checkout process was too long and complicated.\footcite[Cp.][]{Baymard2022}
Additional research shows, that a requirement as simple as entering the e-mail address has a high negative impact on conversion rate\footcite[Cp.][p. 4]{McDowell2016}.

%image of the survey?
Web \ac{SSO} offers solutions to all factors which increase conversion rates mentioned above.
By signing in with an \ac{IdP} before or during the checkout process, users do not have to create a seperate
account with the e-commerce site. Even for web shops that don't require users to create an account, \ac{SSO} offers benefits.
Traditionally users have had to re-enter their personal information like name, address, phone number and birth date
with each site they were shopping on\footcite[Cp. ][p. 24]{Beltran2016}. With \ac{SSO} users can choose to share the information they have
registered with the \ac{IdP} and thus skip the process of entering their information.
Thus, \ac{SSO} increases the checkout speed both with required account creation and without.
This also directly addresses two common reasons why customers abandon their cart during the checkout process.

Although no conclusive research could be found on this, social login might also be able to increase the number of total signups
in a non e-commerce scenario. An example of this is the social media platform Reddit, which measured a 50\% - 60\% increase
in sign ups across their desktop website and Android app after implementing Google's sign in feature\footcite[Cp.][]{GoogleReddit}.

\subsection{Reducing Development Cost}

User account management, authentication and authorization are big parts of web development and require
a lot of development resources. They are also some of the most critical parts of any website,
as security vulnerabilities in these areas can lead to data breaches and hacked accounts.
Research is constantly being conducted in the field of password and database security and there are multiple
standards, which should be adhered to in order to maximize security.\footcite[Cp.][]{Poza2021}

The United States \ac{NIST} publishes a set of digital identity guidelines, which outlines best practices
for website developers. These include standards on implementing authentication forms,
developing password criteria, storing passwords securely, designing appropriate databases and using \ac{MFA}.\footcite[Cp.][]{Poza2021}

With \ac{SSO}, the burden of implementing all these security relevant and therefore critical aspects of a website is put entirely on the \ac{IdP}. 
Furthermore, providers like Google
have an incentive to make the implementation of their \ac{SSO} as simple as possible in order to boost adoption.
The complexity of implementing auth and the ease of integrating \ac{SSO} leads to a reduction in development resources, complexity and cost.\footcite[Cp.][p. 22]{Bazaz2016}

Of course this is only the case if the website completely relies on \ac{SSO} for log ins and doesn't have an additional username and password login option.
In that case complexity can not be reduced, as the auth infrastructure still has to be developed.

\subsection{Reducing Operation Cost}

In addition to development, the administration of a website is simplified when using \ac{SSO}.
User data stored with the \ac{IdP} does not need to be managed by the \ac{SP}, reducing storage needs,
operation complexity and cost. Of course, the degree of savings depends on what percentage of users is willing
to adopt \ac{SSO} and wether or not traditional login is still needed\footcite[Cp.][p. 22]{Bazaz2016}.
As shown in section \ref{sec:sign_in_with_google}, \acp{IdP} like Google make it possible to completely
abandon account management and only use Google's identity management service.
Depending on the needs of the \ac{SP}, this approach might limit the functionality of the website\footcite[Cp.][p. 22]{Bazaz2016}.

\subsection{Increasing User Security}

As users signing in through social login don't have to create passwords for each website they have an account with, security is increased.
On average, users seem to have between six and seven unique passwords. Each of these passwords is reused on just
under six different sites, which totals an average of about 36 unique login combinations that each user has to remember.\footcite[Cp.][p. 4]{Florencio2006}
Reusing the same password on multiple sites is insecure.\footcite[Cp.][p. 249]{Pashalidis2003}
If a website is compromised and passwords are leaked to the public, hackers can try the same username and password
combination on other websites to gain access. Additionally, if a user is subject to a phishing attack,
hackers are able to compromise multiple of the user's accounts.\footcite[Cp.][]{McDade2022}
Even if a website is not responsible for a password leak, users hacked this way might still associate a negative
experience with the brand and reduce interactions in the future.
\ac{SSO} eliminates this issue, as users do not need to think of new passwords for every login and are therefore
not tempted to reuse the same password.

% https://scholar.google.com/scholar?hl=de&as_sdt=0%2C5&q=data+breaches+google+facebook&btnG=
% https://watermark.silverchair.com/tyw003.pdf?token=AQECAHi208BE49Ooan9kkhW_Ercy7Dm3ZL_9Cf3qfKAc485ysgAAAtAwggLMBgkqhkiG9w0BBwagggK9MIICuQIBADCCArIGCSqGSIb3DQEHATAeBglghkgBZQMEAS4wEQQMGWcvk0He3Ahd1tHlAgEQgIICg6VB5sHyFHeMRNmRO8LCIY8UCotYB2j0bW9GoWwkrCDvgNpeYaMhFn8hbH-eNZyokJ6QtNMbluAmcXDFlz5sIfPneb4rrfVTez5gFhS42fnQM0UdyzoChiaZv8BDgM_j0_hOtyjG3UntOVwSmy6BO68hnNLOHyUfCqnMTIhUlQXJ7q4qmA8hiHYI2MiEcwIe9_pwLXGs-Z1UuHV9774UEGINyLLDpItgMMzXN7crp7UhtChyFbaEh3gUAWmwfu8Dclkyb385MLxRsyRDlMmRaHOMfa-XzQdejcs7ZTdZUcyQfwOgiVJ-_6X_TBuhL3EOTBxJupK8HAQ2ckRj1c_ZLFa6n9DY2a5F-TijMHyiy5V6Jrs2-A5luwdhup0XFLqzZezvBNkliSzSTzH65QafTvLiMBuKTHw89qswyprl9rDwXyI9XHFfeEofRNj_EWKQk-d5ph47iN28f09CKVYgqyQwNeuc3F0flbaJP640yUb_2bllOBancf2oV2QIDfGvhGP9mdD1wRMzjOWwa3e9oINuk9ROh9y_GEK4Uw55iUJDNmkDoiZLfGZCMDtyTO7eSJSrLQkwgwOEuwYEBWBKSDZNwGYaIf7rQB_Fsr0DefrtFGm4cYYPCjnZCoIw1TIWwHheoBzkd6iuMPHlDxZAW1UGUD25dSldlhymAvPcJsnyCI0gHnqdjU05f_ZWQpVvn4PVzNwt2-lRibA79KfvKE97_ySpC9-a98JmQCo12tsFnzHce3nTL-DLbwdOp1Xyy2vZEACARCB0xlukb7GPIdQK96tJ9RTO6GtaSwV8m5q0sWSsVUBPChFCqxCSwkBDkULnPAt9vUGhIDtyzw8p2IF6d4U
% https://sci-hub.hkvisa.net/10.1080/15536548.2012.10845654
% https://firewalltimes.com/google-data-breach-timeline/
% https://firewalltimes.com/facebook-data-breach-timeline/
% https://firewalltimes.com/apple-data-breach-timeline/

%When handling authentication on the website, the service provider (owner of the website) is solely responsible for ensuring security to prevent data breaches and leaked passwords.
%When designing a sign in form, developers have to specify certain requirements for user passwords.
%For example, a password should have a certain length and complexity

% https://auth0.com/blog/dont-pass-on-the-new-nist-password-guidelines/


% Increase in Conversion Rate

% file:///home/luisp/Downloads/A_Review_on_Single_Sign_on_Enabling_Technologies_a.pdf
% Increased user productivity
% Increased developer productivity
% Simple administration
% Reduced IT costs
% Foundation for B2B collaboration
% Password Fatigue & likely increase in signups
% Improved data security

% own ideas
% Higher conversion rates due to user convenience
% Sole use of SSO providers for account management
% Lower risk of user credential leaks
% Tracking users across platforms
% Cheap and easy adoption of new security standards