\newpage
\section{Opportunities}

\subsection{Higher Conversion Rates}

The conversion rate is one of the most important e-commerce metrics.
It gives the percentage of users who visited the website that resulted in actual sales.\footcite[Cp.][p. 1]{Gabir2018}
There are a number of factors that can influence conversion rate.
On the one hand these include multiple aspects that have nothing to do with the website.
The type and range of products offered on the website are important.
A store selling clothing might have a much higher conversion rate than a website selling luxury cars, for example\footcite[Cp.][p.165]{Fatta2018}.
The pricing strategy also has an impact. If the website is offering attractive pricing on their products and additional
bonuses like free shipping or frequent discounts, conversion rates are increased\footcite[Cp.][p.165]{Fatta2018}.
The promotion aspect is another important factor. If the brands advertising clearly an effectively promotes its products
visitors are more likely to buy on the first visit\footcite[Cp.][p.165]{Fatta2018}.

The factors mentioned above are related to the web presence, in that offers like discounts and free shipping
need to be effectively presented on the platform. But in order to have a decent conversion rate, technical
factors on the website also have to be addressed\footcite[Cp.][p. 5]{Gabir2018}.

Usability is directly related to how the website is built and which features the e-commerce system has\footcite[Cp.][p. 2f]{Gabir2018}.
There are three dimensions of usability on a website.
Information Quality describes how accurate, relevant and timely the content on the website is.
Service Quality describes how interactive and responsive the website is, how good the search function works
and how well the security and privacy policies are presented.
System Quality includes how easy the site is to navigate, how fast the checkout process is
and aspects like accessibility, and consistency of layout.\footcite[Cp. ][p. 3]{Kuan}
Research has shown, that all three dimensions have a direct impact on conversion rates and customer retention.
Users, which perceive a higher information and system quality are significantly more likely to complete
a purchase\footcite[Cp.][p. 6]{Kuan}.
Service quality might have a measurable impact, but studies diverge here\footcite[Cp.][p. 6]{Kuan}\footcite[Cp.][p. 5]{Gabir2018}.
A survey shows that the second most common reason why adults in the \ac{US} do not complete the checkout
process, is that they had to create an account. 17\% said, that the checkout process was too long and complicated.\footcite[Cp.][]{Baymard2022}
Additional research shows, that a requirement as simple as entering the e-mail address has a high negative impact on conversion rate\footcite[Cp.][p. 4]{McDowell2016}

%image of the survey?
Web \ac{SSO} offers solutions to all factors which increase conversion rates mentioned above.
By signing in with an \ac{IdP} before or during the checkout process, users do not have to create a seperate
account with the e-commerce site. Even for web shops, that don't require users to create an account, \ac{SSO} offers benefits.
Traditionally users have had to re-enter their personal information like name, address, phone number and birth date
with each site they were shopping on\footcite[Cp. ][p. 24]{Beltran2016}. With \ac{SSO} users can choose to share their information that they have
registered with the \ac{IdP} and thus skip the process of entering their information.
Thus, \ac{SSO} increases the checkout speed both with required account creation and without.
This also directly addresses two common reasons why customers abandon their cart during the checkout process.

\subsection{Simpler Account Management}
%Only using SSO providers and thus not having to store passwords...

User account management, authentication and authorization are big parts of web development and require
a lot of development resources.
Using \ac{SSO}, the burden of implementing user authentication forms, developing password criteria,
storing passwords securely, designing appropriate databases and many more aspects are completely put
on the \ac{IdP}. Furthermore, providers like Google and the well defined \ac{SSO} protocols
make implementation of social login services easy for developers.
This helps to reduce development resources, complexity and cost.\footcite[Cp.][p. 22]{Bazaz2016}

In addition to development, the administration of a website is simplified when using \ac{SSO}.
User data stored with the \ac{IdP} does not need to be managed by the \ac{SP}, reducing storage needs
operation complexity and cost. Of course, the degree of savings depends on what percentage of users is willing
to adopt \ac{SSO} and wether or not traditional login is still needed\footcite[Cp.][p. 22]{Bazaz2016}.
As shown in section \ref{sec:sign_in_with_google}, \acp{IdP} like Google make it possible to completely
abandon account management and only use Google's identity management service.
Depending on the needs of the \ac{SP}, this approach might limit the functionality of the website\footcite[Cp.][p. 22]{Bazaz2016}.

% Implementing MFA complex

\subsection{Increased Security}

As users that use social login don't have to create passwords for each website they have an account with, security is increased.
On average, users seem to have between six and seven unique passwords. Each of these passwords is reused on just
under six different sites, which totals an average of about 36 unique login combinations that each user has to remember.\footcite[Cp.][p. 4]{Florencio2006}
Reusing the same password on multiple sites is very insecure.
If a website is compromised and passwords are leaked to the public, hackers can try the same username and password
combination on other websites to gain access. Additionally, if a user is subject to a phishing attack,
hackers are able to compromise multiple of the user's accounts.\footcite[Cp.][]{McDade2022}
Even if a website is not responsible for a password leak, users hacked this way will still associate a negative
experience with the brand and reduce interactions in the future.
\ac{SSO} eliminates this issue, as users do not need to think of new passwords for every login and are therefore
not able to reuse the same password twice.
As \acp{IdP} are usually large corporations, their security standards regarding credentials are strict and
breaches are less likely to happen. \textbf{(source)}

%
% https://scholar.google.com/scholar?hl=de&as_sdt=0%2C5&q=data+breaches+google+facebook&btnG=
% https://watermark.silverchair.com/tyw003.pdf?token=AQECAHi208BE49Ooan9kkhW_Ercy7Dm3ZL_9Cf3qfKAc485ysgAAAtAwggLMBgkqhkiG9w0BBwagggK9MIICuQIBADCCArIGCSqGSIb3DQEHATAeBglghkgBZQMEAS4wEQQMGWcvk0He3Ahd1tHlAgEQgIICg6VB5sHyFHeMRNmRO8LCIY8UCotYB2j0bW9GoWwkrCDvgNpeYaMhFn8hbH-eNZyokJ6QtNMbluAmcXDFlz5sIfPneb4rrfVTez5gFhS42fnQM0UdyzoChiaZv8BDgM_j0_hOtyjG3UntOVwSmy6BO68hnNLOHyUfCqnMTIhUlQXJ7q4qmA8hiHYI2MiEcwIe9_pwLXGs-Z1UuHV9774UEGINyLLDpItgMMzXN7crp7UhtChyFbaEh3gUAWmwfu8Dclkyb385MLxRsyRDlMmRaHOMfa-XzQdejcs7ZTdZUcyQfwOgiVJ-_6X_TBuhL3EOTBxJupK8HAQ2ckRj1c_ZLFa6n9DY2a5F-TijMHyiy5V6Jrs2-A5luwdhup0XFLqzZezvBNkliSzSTzH65QafTvLiMBuKTHw89qswyprl9rDwXyI9XHFfeEofRNj_EWKQk-d5ph47iN28f09CKVYgqyQwNeuc3F0flbaJP640yUb_2bllOBancf2oV2QIDfGvhGP9mdD1wRMzjOWwa3e9oINuk9ROh9y_GEK4Uw55iUJDNmkDoiZLfGZCMDtyTO7eSJSrLQkwgwOEuwYEBWBKSDZNwGYaIf7rQB_Fsr0DefrtFGm4cYYPCjnZCoIw1TIWwHheoBzkd6iuMPHlDxZAW1UGUD25dSldlhymAvPcJsnyCI0gHnqdjU05f_ZWQpVvn4PVzNwt2-lRibA79KfvKE97_ySpC9-a98JmQCo12tsFnzHce3nTL-DLbwdOp1Xyy2vZEACARCB0xlukb7GPIdQK96tJ9RTO6GtaSwV8m5q0sWSsVUBPChFCqxCSwkBDkULnPAt9vUGhIDtyzw8p2IF6d4U
% https://sci-hub.hkvisa.net/10.1080/15536548.2012.10845654
% https://firewalltimes.com/google-data-breach-timeline/
% https://firewalltimes.com/facebook-data-breach-timeline/
% https://firewalltimes.com/apple-data-breach-timeline/

% Implementing MFA complex

\subsection{Cross-Platform User Tracking}