\section{Introduction}
% Could also be Motivation & Problem Definition, Goal and Approach
% Or Motivation and Methodology
\subsection{Abstract}

\subsection{Motivation}

One of the most difficult aspects of operating a website or app is dealing with user
authentication and passwords.

Regularly, hackers try to break into the databases of internet services to reveal account data and 
passwords of their users.

The best-practices for encrypting passwords and storing them safely are changing
continuously and keeping up with new security standards and requirements is costly and time consuming.

Managing passwords and usernames is not only a technical challenge, but also puts a burden
on end users.\footcite[Cp. ][p. 459]{Hoonakker2009}

A majority of users still don't manage their passwords in the recommended way, which is
generating a new password for each login and storing every password in a secure password safe.
A modern internet user on average uses x different services that require a login.
This results in most users using the same password for many or all of these services,
which makes it easy for hackers to overtake a users online presence if only one of their
passwords is hacked or leaked.

Because of these problems, SSO is good fivehead.







Digital services have revolutionized the way people 


Remembering passwords is hard bla bla

Reusing the same password is insecure and hackers have used this



- Account management is hard
- Hackers are trying to find security flaws and publish passwords
- Two factor authentication is annoying for customers etc.



Das Geschäftliche Umfeld moderner Unternehmen ist wechselhaft und komplex.
Kundenanforderungen sowie Marktbedingungen verändern sich stetig und der Hohe Grad an
Vernetzung zwischen verschiedenen Anwendungen erschwert die Übersicht.
Aufgrund dieser schweren Bedingungen, benötigen große Softwareunternehmen wie
Microsoft oder Alphabet Möglichkeiten, schnell auf sich ändernde Bedingungen zu reagieren,
Kundenanforderungen rasant umzusetzen und die Zusammenarbeit und Wissensverteilung effizient
zu organisieren. In dieser Arbeit wird untersucht, ob das Konzept DevOps plausible
Chancen bietet, um diesen hohen Anforderungen gerecht zu werden.

\subsection{Problem Definition and Goal}

Ziel dieser Arbeit ist es, die Prinzipien des DevOps Konzepts im Software Engineering
zu beschreiben, sowie seine Chancen für Unternehmen und Entwicklungsteams darzulegen.
Zudem soll DevSecOps thematisch eingeordnet werden. 

\subsection{Approach}

Die Arbeit beginnt mit einer Darstellung und Erklärung des Software Lebenszyklus und seiner
Schritte. Anschließend wird das Wasserfallmodell als Beispiel für ein konventionelles Ablaufmodell
erläutert. Agile Entwicklung wird abgegrenzt und anhand von Scrum erklärt.
Im Hauptteil wird zunächst der Begriff DevOps definiert und Ursprung und Ziel des Konzepts
erläutert. Anschließend werden die Grundsätze von DevOps anhand der sechs Säulen dargelegt.
Nachdem die Prinzipien von DevOps klar sind, werden Chancen gegeben, die DevOps modernen
Softwareunternehmen bietet. Daraufhin wird der Begriff DevSecOps im Kontext von DevOps eingeordnet.
Abschließend werden die Erkenntnisse zusammengefasst und mögliche Kritik am Konzept adressiert.