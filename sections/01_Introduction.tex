\section{Introduction}

\subsection{Motivation}

% Password fatigue
The web is a big part of most people's lives. 
Social media services are used to keep in touch with friends and family,
discover new things and keep up to date with ones individual interests.
Streaming services and video platforms are used to consume content,
e-commerce sites are used for shopping and
online productivity suites are used constantly in the workplace in a wide
range of industries.

Most modern web services require users to create an account so the platform
can keep track of user data and provide personal recommendations.
This results in the average user having around 36 unique login combinations they need to remember\footcite[Cp.][p. 4]{Florencio2006},
which leads to many users forgetting their password or re-using passwords on multiple sites.

There are many solutions to this problem, such as password managers or auto-fill features in modern web browsers.
A different solution that doesn't rely on password management on the user's side is called Single Sign-On (SSO).
It allows users to log into many different web services using a single set of credentials\footcite[Cp.][134]{Radha2012}.
Large internet corporations like Google and Facebook provide services that allow users to sign into
different websites using their existing account\footcite[Cp.][p. 2]{Gafni2014}.
Websites get the benefit of providing users with a quick way to sign in, simplifying their user interfaces
and getting access to existing user data\footcite[Cp.][]{GoogleSignIn2022}.
This paper examines how \ac{SSO} works, what opportunities it provides and which risks are associated with implementing it.

\subsection{Research Goals \& Structure}

This paper's structure is tightly aligned with its goals.

The first goal is to give an understanding of how \ac{SSO} functionality works.
To achieve this, the basics of authentication and authorization are explained in section \ref{sec:authn_authz}.
Next, important terms related to \ac{SSO} are defined and the types and use cases,
as well as protocols and technologies are examined (section \ref{sec:sso_fim}). Then an overview of \ac{SSO} providers is given.

The second goal is to present opportunities that \ac{SSO} gives businesses that operate websites and platforms
targeting end users on the open web. To achieve this, important opportunities are explained and discussed in section \ref{sec:opportunities}.

The third goal is to present and discuss some of the risks that come with implementing \ac{SSO} from the point of view of a
business operating a website (section \ref{sec:risks}).

Finally, the paper is concluded and possible shortcomings are discussed.