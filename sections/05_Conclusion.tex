\newpage
\section{Conclusion}

In conclusion, Web \ac{SSO} and Social Login are exiting technologies that give \acp{SP} a lot of
additional ways to offer quick and easy sign in functionality to their users.
This paper has shown, that the technology provides operators of web platforms and e-commerce sites
with many opportunities to improve their business metrics, security and usability if they are willing
to manage the risks.
Table \ref{tbl:opps_risks} summarizes the opportunities and risks presented in sections \ref{sec:opportunities} and \ref{sec:risks}.

\begin{table}[H]
    \centering
    \begin{tabular}{llrrr}
        \toprule
        Opportunities & Risks \\
        \midrule
          Increasing Conversion Rates & User Acceptance \\
          Reducing Development Cost & \ac{IdP} as single point of failure \\
          Reducing Operation Cost & Security of \ac{SSO} technologies \\
          Increasing User Security & Security Standards of \acp{IdP} \\
        \bottomrule
        \end{tabular}        
    \caption{Opportunities and Risks of using \ac{SSO} on websites}%
    \label{tbl:opps_risks}
  \end{table} 

It has to be mentioned, that many of the discussed opportunities like reducing development cost are only fully applicable, if
\ac{SSO} completely replaces traditional login with username and password. On the other hand, this greatly
increases risks like dependency on the availability of \acp{IdP}.

Lastly, it is important to understand the limits of this paper. The examined aspects only cover a subset of the possible
risks associated with \ac{SSO}. Focus was put on the Social Login aspect of \ac{SSO} targeted at businesses providing
services to end users on the public web. Mainly the technical and business aspects of \ac{SSO} were focused,
leaving out possible risks in other areas like the legal perspective of Social Login\footcite[Cp.][]{Karegar2018}.
Most of the opportunities outlined in the paper only apply to businesses that develop their own websites.
The possible existence of frameworks or services that might provide simple solutions to account management were
not taken into consideration. All-in-one solutions like Wix or Shopify were also not considered.


% Table summarizing results.
% Only looked at it from a technical perspective.
% Focused on scenario, that website is developed by the company itself.
% Most of the points covered in this paper do not applied to websites hosted using services
% like Shopify or Wix.
% Even if self developed, there might be frameworks that drastically reduce the complexity of
% account management.
% Even then opportunities like the increased conversion rate make a good argument for using Web \ac{SSO}.